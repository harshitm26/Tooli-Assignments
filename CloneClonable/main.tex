%%%%%%%%%%%%%%%%%%%%%%%%%%%%%%%%%%%%%%%%%
%
% 
%
%%%%%%%%%%%%%%%%%%%%%%%%%%%%%%%%%%%%%%%%%

%----------------------------------------------------------------------------------------
%	PACKAGES AND OTHER DOCUMENT CONFIGURATIONS
%----------------------------------------------------------------------------------------

\documentclass[paper=a4, fontsize=11pt]{scrartcl} % A4 paper and 11pt font size
\usepackage[]{algorithm2e} % Used for loading the algorithm package


\usepackage{cite}	%for bibtex
\usepackage{hyperref}	%for crossreferencing inside text
\usepackage{url}	%for urls



\usepackage{listings} % Required for inserting code snippets
\usepackage[usenames,dvipsnames]{color} % Required for specifying custom colors and referring to colors by name

\definecolor{DarkGreen}{rgb}{0.0,0.4,0.0} % Comment color
\definecolor{highlight}{RGB}{255,251,204} % Code highlight color

\lstdefinestyle{Style1}{ % Define a style for your code snippet, multiple definitions can be made if, for example, you wish to insert multiple code snippets using different programming languages into one document
language=Java, % Detects keywords, comments, strings, functions, etc for the language specified
backgroundcolor=\color{highlight}, % Set the background color for the snippet - useful for highlighting
basicstyle=\footnotesize\ttfamily, % The default font size and style of the code
breakatwhitespace=false, % If true, only allows line breaks at white space
breaklines=true, % Automatic line breaking (prevents code from protruding outside the box)
captionpos=b, % Sets the caption position: b for bottom; t for top
commentstyle=\usefont{T1}{pcr}{m}{sl}\color{DarkGreen}, % Style of comments within the code - dark green courier font
deletekeywords={}, % If you want to delete any keywords from the current language separate them by commas
%escapeinside={\%}, % This allows you to escape to LaTeX using the character in the bracket
firstnumber=1, % Line numbers begin at line 1
frame=single, % Frame around the code box, value can be: none, leftline, topline, bottomline, lines, single, shadowbox
frameround=tttt, % Rounds the corners of the frame for the top left, top right, bottom left and bottom right positions
keywordstyle=\color{Blue}\bf, % Functions are bold and blue
morekeywords={}, % Add any functions no included by default here separated by commas
numbers=left, % Location of line numbers, can take the values of: none, left, right
numbersep=10pt, % Distance of line numbers from the code box
numberstyle=\tiny\color{Gray}, % Style used for line numbers
rulecolor=\color{black}, % Frame border color
showstringspaces=false, % Don't put marks in string spaces
showtabs=false, % Display tabs in the code as lines
stepnumber=5, % The step distance between line numbers, i.e. how often will lines be numbered
stringstyle=\color{Purple}, % Strings are purple
tabsize=2, % Number of spaces per tab in the code
}

% Create a command to cleanly insert a snippet with the style above anywhere in the document
\newcommand{\insertcode}[2]{\begin{itemize}\item[]\lstinputlisting[caption=#2,label=#1,style=Style1]{#1}\end{itemize}} % The first argument is the script location/filename and the second is a caption for the listing



\usepackage[a4paper,pdftex]{geometry}	% Use A4 paper margins
\usepackage[english]{babel}
\usepackage{xcolor} % Required for specifying custom colors
\usepackage{fix-cm} % Allows increasing the font size of specific fonts beyond LaTeX default specifications

\setlength{\oddsidemargin}{0mm} % Adjust margins to center the colored title box
\setlength{\evensidemargin}{0mm} % Margins on even pages - only necessary if adding more content to this template

\newcommand{\HRule}[1]{\hfill \rule{0.2\linewidth}{#1}} % Horizontal rule at the bottom of the page, adjust width here

\definecolor{grey}{rgb}{1,0.9,0.6} % Color of the box surrounding the title - these values can be changed to give the box a different color	


%\usepackage{hyperref} %used for embedding url
\usepackage[T1]{fontenc} % Use 8-bit encoding that has 256 glyphs
\usepackage{fourier} % Use the Adobe Utopia font for the document - comment this line to return to the LaTeX default
\usepackage[english]{babel} % English language/hyphenation
\usepackage{amsmath,amsfonts,amsthm} % Math packages

\usepackage{lipsum} % Used for inserting dummy 'Lorem ipsum' text into the template

\usepackage{sectsty} % Allows customizing section commands
\allsectionsfont{\centering \normalfont\scshape} % Make all sections centered, the default font and small caps

\usepackage{fancyhdr} % Custom headers and footers
\pagestyle{fancyplain} % Makes all pages in the document conform to the custom headers and footers
\fancyhead{} % No page header - if you want one, create it in the same way as the footers below
\fancyfoot[L]{} % Empty left footer
\fancyfoot[C]{} % Empty center footer
\fancyfoot[R]{\thepage} % Page numbering for right footer
\renewcommand{\headrulewidth}{0pt} % Remove header underlines
\renewcommand{\footrulewidth}{0pt} % Remove footer underlines
\setlength{\headheight}{13.6pt} % Customize the height of the header

\numberwithin{equation}{section} % Number equations within sections (i.e. 1.1, 1.2, 2.1, 2.2 instead of 1, 2, 3, 4)
\numberwithin{figure}{section} % Number figures within sections (i.e. 1.1, 1.2, 2.1, 2.2 instead of 1, 2, 3, 4)
\numberwithin{table}{section} % Number tables within sections (i.e. 1.1, 1.2, 2.1, 2.2 instead of 1, 2, 3, 4)

\setlength\parindent{0pt} % Removes all indentation from paragraphs - comment this line for an assignment with lots of text



\date{\normalsize\today} % Today's date or a custom date

\begin{document}
%\maketitle
\thispagestyle{empty} % Remove page numbering on this page

%----------------------------------------------------------------------------------------
%	TITLE SECTION
%----------------------------------------------------------------------------------------

\colorbox{grey}{
	\parbox[t]{1.0\linewidth}{
		\begin{center}
		\fontsize{50pt}{80pt}\selectfont % The first argument for fontsize is the font size of the text and the second is the line spacing - you may need to play with these for your particular title
		\vspace*{0.7cm} % Space between the start of the title and the top of the grey box
		
		\hfill Java Extension:\\
		\hfill Automatic \\
		\hfill Type Inference\\
%		\hfill Clonable Interface\par
		\vspace{0.5cm}
		\end{center}				
		\raggedleft
		\fontsize{25pt}{12pt}\selectfont
		CS698Y Project, 2013-14 II
		\vspace*{0.7cm} % Space between the end of the title and the bottom of the grey box
		
	}
}

%----------------------------------------------------------------------------------------

\vfill % Space between the title box and author information

%----------------------------------------------------------------------------------------
%	AUTHOR NAME AND INFORMATION SECTION
%----------------------------------------------------------------------------------------

{\centering \large 
\hfill Abhimanyu Jaju \   \{\texttt{10327009, abhijaju@iitk.ac.in}\}\\
\hfill Harshit Maheshwari\ \{\texttt{10327290, harshitm@iitk.ac.in}\}\\
\hfill Vinit Kataria\ \{ \texttt{10327807, vinitk@iitk.ac.in}\}\\
\vspace{0.5cm}

\hfill Indian Institute of Technology, Kanpur \\
\hfill Computer Science and Engineering\\
%\hfill \texttt{harshitm@iitk.ac.in} \\



\HRule{1pt}} % Horizontal line, thickness changed here

%----------------------------------------------------------------------------------------

\clearpage % Whitespace to the end of the page

\section{Introduction \label{introduction}}
 When we want classes to support copying functionality we want to `clone` the objects of the class. In Java all classes are extended from `Object` class. `Object` class implements cloneable interface. The `cloneable` interface of java is an empty interface which contains no members at all. It is used by a class to indicate that the class supports cloning. By `supports` it means that the class implements clone() method. However, the clone() method of the `Object` class does shallow copying. If we want to do deep copying we need to do that in the child class on the copy inherited from `Object` class clone method. Some of the basic properties a clone method should satisfy are: 
 \begin{itemize}
 	\item The cloned object does not have the same 		reference as that of original object .ie, o.clone() != o should be true
	\item The cloned object has the same values as that of the original object for all the fields of the class.
ie, o.clone.equals(o) should be true.
	\item The cloned object and the original object should belong to the same class.
ie, o.clone.getClass() == o.getClass() should be true. 
 \end{itemize}
 
\section{Modifier \label{modifier}}
To define clone the clone method of `Object` class has to be overridden. The `clone` method implemented by `Object` class is \textit{protected}.The child class can extend `clone` method using \textit{public/protected} modifier. We cannot use \textit{private/package protected} because it reduces the visibility of inherited method and gives compilation error. 


%\insertcode{"Scripts/code1.java"}{Automatic type-deduction for variables} 
\insertcode{"Scripts/code2.java"}{} 
\insertcode{"Scripts/code3.java"}{Automatic return type-deduction for functions}
\insertcode{"Scripts/code1.java"}{Automatic type-deduction for variables} 
\insertcode{"Scripts/code2.java"}{} 
\insertcode{"Scripts/code3.java"}{Automatic return type-deduction for functions}
\section{`clone` method is necessary \label{cloneMethod}}
It is necessary to implement the clone method along with `cloneable` interface. 
\insertcode{"Scripts/cloneMethod.java"}{`Clone method not implemented in child class`\label{code:cloneMethod}}
The code \nameref{code:cloneMethod} gives the following exception \\
\textit{Unresolved Compilation Problem\\	The method clone() from the type Object is not visible
}
\section{`CloneNotSupportedException` \label{cloneNotSupportedException}\cite{Suresh, Oracle}}
When creating objects with `clone` method we need to extend the cloneable interface. Otherwise the compiler gives `CloneNotSupportedException`. Note, that `CloneNotSupported` is a checked exception and therefore, we use keyword `throws` in clone method when we use the clone method of the parent class.
% Detailed explanation is given in Section \nameref{section:inheritance}. 
\insertcode{"Scripts/code2.java"}{CloneNotSupportedException Example}
Above code gives the following exception:\\
\textit{
	Unhandled exception type CloneNotSupportedException\\
	at Animal.clone(Animal.java:5)
}
\section{Shallow Cloning vs Deep Cloning\label{shallowDeep}\cite{noAuthor1, Joe}}
The `object` class clone method by default returns shallow copy of the object. In order to create deep copy we need to override the `clone` method of the parent class.



\subsection{Shallow Cloning \label{shallow}}
As mentioned in Section \ref{introduction} `clone` method of `Object` class returns a shallow copy.
\insertcode{"Scripts/shallow2.java"}{Shallow Cloning\label{code:shallowCode}}
\insertcode{"Scripts/shallow2.output"}{Shallow Cloning Output\label{output:shallowCodeOutput}}


 As we can see from code \nameref{code:shallowCode} and the output \nameref{output:shallowCodeOutput} if the members of child class consists of 
\begin{itemize}
	\item \textbf{Primitive/Immutable data types}\\
	New copies are created. The value of primitive type `i` is copied in `cClone`.
	\item \textbf{User Defined classes}\\
	Only references are copied. Only the reference of field `StringBuffer b` is copied in clone `dClone`. 
\end{itemize}
The desired behaviour was to have different values in \textit{d.b} and \textit{dClone.b}. 
\subsection{Deep Cloning}
For deep cloning, we must satisfy following:\
\begin{itemize}
	\item All the members of the class  should implement Cloneable interface and should override Object's clone() method.
	\item If a member method does not follow the above rule, we must create a new instance of the member class and copy all the attributes to the new object(ensuring deep copying).
\end{itemize}

In order to implement the desired behaviour we need to modify the clone method of the child class as shown in Code \nameref{code:deepCode}
\insertcode{"Scripts/deep.java"}{`Deep Cloning`\label{code:deepCode}}
\insertcode{"Scripts/deep.output"}{`Deep Cloning Output` \label{output:deepCodeOutput}}

\section{Inheritance \label{section:inheritance}}
\subsection{Inheritance from `Object` Class\label{subsection:inheritanceObject}}
	If we use `clone` method of Object class then then the clone method should throw \nameref{output:inheritanceOutput} exception. 
	

\subsection{Inheritance from parent class\label{subsection:inheritanceParent}}
	If we use `clone` method of parent class and if parent class returns a new object then we get \textit{ClassCastException}	
\insertcode{"Scripts/code5.java"}{`ClassCastException Example` \label{code:inheritance}}
\insertcode{"Scripts/inheritanceOutput.output"}{`ClassCastException` \label{output:inheritanceOutput}}
So, if a class inherits a base class, then the base class must make a call to super.clone() in order to invoke Object.clone() method. 
The output of code \nameref{code:typeCastingError} is \nameref{output:typeCastingErrorOutput}
\insertcode{"Scripts/typeCastingError.java"}{`TypeCastingError` \label{code:typeCastingError}}
\insertcode{"Scripts/typeCastingError.output"}{`TypeCastingError output`\label{output:typeCastingErrorOutput}}
We get error because after dynamic type inferencing the clone() method of Dog() is called and which in turn calls clone() method of Animal class which returns a new object of Animal class which cannot be typecasted to Dog.
\subsection{Typecasting in Cloning\label{subsection:typecasting}}
Consider the code \nameref{code:typecasting}
\insertcode{"Scripts/typecasting.java"}{`TypeCasting in Cloning`\label{code:typecasting}}
In this we see we can convert from cloned `Dog` to `Animal` and then back to `Animal`.

\subsection{Cloneable Interface in parent class\label{subsection:cloneableParent}}
Consider the code \nameref{code:cloneableParent}
\insertcode{"Scripts/cloneableParent.java"}{`Parent class need not implement Cloneable interface` \label{code:cloneableParent}}
The code runs without any errors. 
\section{Acknowledgements}
Prabhat Pandey (10498) and I, thoroughly discussed the assignment before we started making the report.

\bibliographystyle{plain}
\bibliography{references}
\end{document}