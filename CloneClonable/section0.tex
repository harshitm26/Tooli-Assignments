\section{Introduction \label{introduction}}
 When we want classes to support copying functionality we want to `clone` the objects of the class. In Java all classes are extended from `Object` class. `Object` class implements cloneable interface. The `cloneable` interface of java is an empty interface which contains no members at all. It is used by a class to indicate that the class supports cloning. By `supports` it means that the class implements clone() method. However, the clone() method of the `Object` class does shallow copying. If we want to do deep copying we need to do that in the child class on the copy inherited from `Object` class clone method. Some of the basic properties a clone method should satisfy are: 
 \begin{itemize}
 	\item The cloned object does not have the same 		reference as that of original object .ie, o.clone() != o should be true
	\item The cloned object has the same values as that of the original object for all the fields of the class.
ie, o.clone.equals(o) should be true.
	\item The cloned object and the original object should belong to the same class.
ie, o.clone.getClass() == o.getClass() should be true. 
 \end{itemize}
 